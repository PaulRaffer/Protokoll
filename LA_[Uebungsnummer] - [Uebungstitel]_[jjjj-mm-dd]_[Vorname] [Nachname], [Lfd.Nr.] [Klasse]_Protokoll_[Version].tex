%%%%%%%%%%%%%%%%%%%%%%%%%%%%%%%%%%%%%%%%%%%%%%%%%%%%%%%%%%%%%%%%
% Filename:           LA_[Uebungsnummer] - [Uebungstitel]_[jjjj-mm-dd]_[Vorname] [Nachname], [Lfd.Nr.] [Klasse]_Protokoll_[Version].tex
% Autor:              Paul Raffer                              %
% Zuletzt bearbeitet: 2020-10-04                               %
% Beschreibung:       In diesem File werden die benötigten     %
%                     Packages und die anderen LaTeX-Files     %
%                     (macros.tex, titlepage.tex, content.tex) %
%                     inkludiert.                              %
%                                                              %
%                     DIESES FILE MUSS IM NORMALFALL NICHT     %
%                     BEARBEITET WERDEN!!!                     %
%                                                              %
%                     DIESES FILE MUSS KOMPILIERT WERDEN UM    %
%                     DAS FERTIGE PROTOKOLL ALS OUTPUT ZU      %
%                     ERHALTEN!                                %
%%%%%%%%%%%%%%%%%%%%%%%%%%%%%%%%%%%%%%%%%%%%%%%%%%%%%%%%%%%%%%%%

\documentclass[a4paper]{article}

\usepackage{lastpage}
\usepackage[ngerman]{babel}
\usepackage[utf8]{inputenc}
\usepackage{lmodern}
\usepackage{setspace}
\usepackage{fancyhdr}
\usepackage[a4paper,margin=2.5cm]{geometry}
\usepackage{tabularx}
\usepackage{graphicx}
\usepackage{pdfpages}
\usepackage{listings}
\usepackage{color}
\usepackage{amssymb}
%\usepackage{extarrows}
\usepackage{float}

\usepackage{dcolumn}
\newcolumntype{d}[1]{D{.}{.}{#1}}

\usepackage{array}
\newcolumntype{$}{>{\global\let\currentrowstyle\relax}}
\newcolumntype{^}{>{\currentrowstyle}}
\newcommand{\rowstyle}[1]{\gdef\currentrowstyle{#1}%
	#1\ignorespaces
}

\renewcommand{\familydefault}{\sfdefault}

\lstdefinelanguage
	[CortexM3]{Assembler}
	[x86masm]{Assembler}
	{morekeywords={
		AREA,READONLY,EXPORT,DCB,LDR,BEQ,CBNZ,CBZ,BFI,BNE,B,ORR,ALIGN,BIC,UDIV,STRH,DCD,TST,LDRB,
	}}
	
\lstset{
	tabsize=4,
	keywordstyle=\color{green},
	commentstyle=\color{gray},
	stringstyle=\color{blue},
}

\pagestyle{fancy}
\fancyhf{}

%%%%%%%%%%%%%%%%%%%%%%%%%%%%%%%%%%%%%%%%%%%%%%%%%%%%%%%%%%%%%%%%

%%%%%%%%%%%%%%%%%%%%%%%%%%%%%%%%%%%%%%%%%%%%%%%%%%%%%%%%%%%%%%%%
% Filename:           macros.tex                               %
% Autor:              Paul Raffer                              %
% Zuletzt bearbeitet: 2020-10-04                               %
% Beschreibung:       In diesem File können die Werte von der  %
%                     Makros festgelegt werden.                %
%%%%%%%%%%%%%%%%%%%%%%%%%%%%%%%%%%%%%%%%%%%%%%%%%%%%%%%%%%%%%%%%

\newcommand{\schulekurzname}             {<Abkürzung der Schule>}
\newcommand{\schulelangname}             {<Name der Schule>}
\newcommand{\schuleabteilung}            {<Abteilung>}
\newcommand{\schulelogo}                 {schule_logo.png}   % Schullogo unter "./res/img/schule_logo.pdf" abspeichern


\newcommand{\uebungstitel}               {<Übungstitel laut Laborplan>}
\newcommand{\uebungsnummer}              {<Übungsnummer>}
\newcommand{\uebungsleiter}              {<Name Übungsleiter>}
\newcommand{\labor}                      {<Labor>}
\newcommand{\nummerderabnahmeeinheit}    {<Nr. Abnahmeeinheit>}
\newcommand{\uebungsdatum}               {<jjjj-mm-dd>}
\newcommand{\abgabedatum}                {<jjjj-mm-dd>}
\newcommand{\klasse}                     {<xABHEL>}
\newcommand{\gruppe}                     {<?a/?b>}
\newcommand{\uebungsteilnehmerIvorname}  {<Vname1>} \newcommand{\uebungsteilnehmerInachname}  {<Nname1>} \newcommand{\uebungsteilnehmerI}  {\uebungsteilnehmerIvorname\ \uebungsteilnehmerInachname}
\newcommand{\uebungsteilnehmerIIvorname} {<Vname2>} \newcommand{\uebungsteilnehmerIInachname} {<Nname2>} \newcommand{\uebungsteilnehmerII} {\uebungsteilnehmerIIvorname\ \uebungsteilnehmerIInachname}
\newcommand{\uebungsteilnehmerIIIvorname}{<Vname3>} \newcommand{\uebungsteilnehmerIIInachname}{<Nname3>} \newcommand{\uebungsteilnehmerIII}{\uebungsteilnehmerIIIvorname\ \uebungsteilnehmerIIInachname}
\newcommand{\uebungsteilnehmerIVvorname} {<Vname4>} \newcommand{\uebungsteilnehmerIVnachname} {<Nname4>} \newcommand{\uebungsteilnehmerIV} {\uebungsteilnehmerIVvorname\ \uebungsteilnehmerIVnachname}
\newcommand{\schriftfuehrervorname}      {<VnameX>} \newcommand{\schriftfuehrernachname}      {<NnameX>} \newcommand{\schriftfuehrer}      {\schriftfuehrervorname\ \schriftfuehrernachname}
\newcommand{\abwesendeuebungsteilnehmer} { & keiner \\ }     % Abwesende Übungsteilnehmer in LaTeX-Tabellenform (& = nächste Spalte; \\ = nächste Zeile) eintragen, wobei die erste Spalte immer leer bleibt und in der zweiten Spalte ein Name eines abwesenden Schülers pro Zeile eingetragen wird
\newcommand{\schueler}                   {\schriftfuehrer}   % Zeile nicht verändern, außer du bist nicht der Schriftführer (z.B. Ersatzprotokoll)


\newcommand{\messprotokoll}              {Messprotokoll.pdf} % Messprotokoll einscannen und unter "./res/doc/Messprotokoll.pdf" abspeichern ODER diese Zeile entfernen und das originale Messprotokoll nach dem Ausdrucken zum Protokoll dazuheften
\newcommand{\inventarliste}              {Inventarliste.pdf} % Inventarliste unter "./res/doc/Inventarliste.pdf" abspeichern


\lhead{\labor}                                             \rhead{\uebungstitel}
\lfoot{\schulekurzname} \cfoot{\schueler\ / \klasse} \rfoot{Seite \thepage\ von \pageref{LastPage}}

\author{\schriftfuehrer}


\begin{document}
	%%%%%%%%%%%%%%%%%%%%%%%%%%%%%%%%%%%%%%%%%%%%%%%%%%%%%%%%%%%%%%%%
% Filename:           titlepage.tex                            %
% Autor:              Paul Raffer                              %
% Zuletzt bearbeitet: 2020-10-04                               %
% Beschreibung:       erste Seite des Protokolls, Algemeiner   %
%                     Teil und Inhaltsverzeichnis              %
%                                                              %
%                     DIESES FILE MUSS IM NORMALFALL NICHT     %
%                     BEARBEITET WERDEN!!!                     %
%%%%%%%%%%%%%%%%%%%%%%%%%%%%%%%%%%%%%%%%%%%%%%%%%%%%%%%%%%%%%%%%

\begin{titlepage}

	\begin{flushright}
		\includegraphics[width=2cm]{res/img/\schulelogo}
	\end{flushright}

	\begin{center}
		\begin{spacing}{3}
		{\LARGE{\bfseries{\schulelangname}}}
		\end{spacing}
		{\LARGE \schuleabteilung}
	\end{center}

	\begin{table}[h]
		\renewcommand{\arraystretch}{1.4}
		\begin{tabularx}{\textwidth}{|$X@{\hspace{7mm}}|^X|^X|}
			\hline
			Klasse / Jahrgang:                           & Gruppe:                    & Übungsleiter:                                                                                                                       \\
			\rowstyle{\large} \raggedleft \klasse        & \centering \gruppe         & \hspace{7mm}\uebungsleiter                                                                                                          \\
			\hline
			Übungsnummer:                                & \multicolumn{2}{l|}{Übungstitel:}                                                                                                                                \\
			\rowstyle{\large} \raggedleft \uebungsnummer & \multicolumn{2}{l|}{\hspace{2cm}{\large \uebungstitel}}                                                                                                          \\
			\hline
			Datum der Übung:                             & \multicolumn{2}{l|}{Teilnehmer:}                                                                                                                                 \\
			\rowstyle{\large} \raggedleft \uebungsdatum  & \multicolumn{2}{l|}{\hspace{2cm}{\large \uebungsteilnehmerInachname, \uebungsteilnehmerIInachname, \uebungsteilnehmerIIInachname, \uebungsteilnehmerIVnachname}} \\
			\hline
			Datum der Abgabe:                            & Schriftführer:             & Unterschrift:                                                                                                                       \\
			\rowstyle{\large} \raggedleft \abgabedatum   & \centering \schriftfuehrer &                                                                                                                                     \\
			\hline
		\end{tabularx}
	\end{table}


	\begin{table}[h]
		\renewcommand{\arraystretch}{1.3}
		\raggedleft
		{\large
		\begin{tabular}{l| p{0.6cm} |l}
			\multicolumn{2}{l|}{}            & Beurteilung \\\hline
			Deckbl., Inhaltsverz.          & &             \\\hline
			Aufgabenstellung               & &             \\\hline
			Dokumentation                  & &             \\\hline
			\hspace{5mm}Messschaltungen    & &             \\\hline
			\hspace{5mm}Messtabellen       & &             \\\hline
			\hspace{5mm}Berechnungen       & &             \\\hline
			\hspace{5mm}Programmlistings   & &             \\\hline
			Auswerung                      & &             \\\hline
			\hspace{5mm}Diagramme          & &             \\\hline
			\hspace{5mm}Berechnungen       & &             \\\hline
			\hspace{5mm}Simulationen       & &             \\\hline
			\hspace{5mm}Schlussfolgerungen & &             \\\hline
			\hspace{5mm}Kommentare         & &             \\\hline
			Inventarliste                  & &             \\\hline
			Messprotokoll                  & &             \\\hline
			Form                           & &             \\\hline
			\multicolumn{2}{l|}{Summe}       &             \\
		\end{tabular}
		}
	\end{table}
\end{titlepage}

\section*{Allgemeiner Teil}

\begin{table}[h]
	{\large
	\begin{tabular}{ll}
		Titel der Übung:            & \uebungstitel                      \\
		Übungsleiter:               & \uebungsleiter                     \\
		Übungsnummer:               & \uebungsnummer                     \\
		Übungsplatz:                & \labor\ / \nummerderabnahmeeinheit \\
		Datum der Übung:            & \uebungsdatum                      \\
		Klasse:                     & \klasse                            \\
		Schriftführer:              & \schriftfuehrer                    \\
		Übungsteilnehmer:           & \uebungsteilnehmerI                \\
									& \uebungsteilnehmerII               \\
									& \uebungsteilnehmerIII              \\
									& \uebungsteilnehmerIV               \\
		Abwesende Übungsteilnehmer: \abwesendeuebungsteilnehmer

	\end{tabular}
	}
\end{table}

\tableofcontents

\newpage

	%%%%%%%%%%%%%%%%%%%%%%%%%%%%%%%%%%%%%%%%%%%%%%%%%%%%%%%%%%%%%%%%
% Filename:           content.tex                              %
% Autor:              Paul Raffer                              %
% Zuletzt bearbeitet: 2020-10-04                               %
% Beschreibung:       In dieses File wird der eigentliche      %
%                     Inhalt des Protokolls geschrieben.       %
%%%%%%%%%%%%%%%%%%%%%%%%%%%%%%%%%%%%%%%%%%%%%%%%%%%%%%%%%%%%%%%%

\section{Theorie zur Übung}
	<kurzer Theorieteil - Hinweis auf Besonderheiten>
	

\section{Titel - Übungsaufgabe 1}

	\subsection{Aufgabenstellung}
		<Gegeben ist, Gesucht ist, als Text angeben>
		
	\subsection{Berechnung}
		<Dimensionierung/Berechnung Schaltung, Simulationen, Messbereichsabschätzungen>
		
	\subsection{Messschaltung}
		<Schaltung, eingestellte Messbereiche, BMKZ -> Zuordnung zu Inventarliste>
		
	\subsection{Messvorgang}
		<was ist zu beachten, was/wie wird eingestellt, Einschalt-Reichenfolge, Ausschalt-Reihenfolge>
		
	\subsection{Messtabelle}
		<Messtabelle, Bezeichnung der Mess- Einstellgrößen laut Messschaltung, Einheiten, verwendete Formeln, Beispielrechnung mit Einheiten>
		
	\subsection{Diagramm}
		<Grafische Auswertung, Kommentare zu Besonderheiten/Abweichungen zur Theorie-Erwartung>
		
	\subsection{Ergebnis}
		<verbales Ergebnis / verbale Erkenntnis aus Übung 1>
		
	
\section{Abschließende Bemerkungen}
	<Besondere Vorkommmisse während der Übung (z.B.: Feueralarm, NOT-AUS), besondere Erkenntnisse, Verbesserungsvorschläge>
	

% ACHTUNG: FÜR DEN ANHANG KÖNNEN VON LaTeX LEIDER NICHT DIE RICHTIGEN SEITENZAHLEN ERMITTELT WERDEN!
% DESHALB MÜSSEN DIESE NACH DEM KOMPILIEREN MANUELL IN DER DATEI
% "LA_[Uebungsnummer] - [Uebungstitel]_[jjjj-mm-dd]_[Vorname] [Nachname], [Lfd.Nr.] [Klasse]_Protokoll_[Version].toc"
% ANGEPASST WERDEN! (Dazu muss der Wert in den letzten geschwungenen Klammern der Zeile verändert werden)
% ANSCHLIESSEND MUSS DIE DATEI GESPEICHERT UND DAS PROTOKOLL NEU KOMPILIERT WERDEN

\addcontentsline{toc}{section}{Anhang}                                          % Die section Anhang wird zum Inhaltsverzeichnis hinzugefügt
	\addcontentsline{toc}{subsection}{Datenblatt XY}                            % Wenn keine Datenblätter benötigt werden kann diese Zeile entfernt werden
	\addcontentsline{toc}{subsection}{Messprotokoll (Mitschrift bei der Übung)}
	\addcontentsline{toc}{subsection}{Inventarliste}

%\includepdf[pages={<Seitenzahlen>}]{res/doc/Datenblätter/XY.pdf} % Nur wenn ein Datenblatt eingebunden werden soll
\includepdf[pages=-]{res/doc/\messprotokoll} % Wenn das original Messprotokoll nach dem Ausdrucken dazugeheftet wird MUSS diese Zeile entfernt werden
\includepdf[pages=-]{res/doc/\inventarliste}

\end{document}
