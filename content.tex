%%%%%%%%%%%%%%%%%%%%%%%%%%%%%%%%%%%%%%%%%%%%%%%%%%%%%%%%%%%%%%%%
% Filename:           content.tex                              %
% Autor:              Paul Raffer                              %
% Zuletzt bearbeitet: 2020-10-04                               %
% Beschreibung:       In dieses File wird der eigentliche      %
%                     Inhalt des Protokolls geschrieben.       %
%%%%%%%%%%%%%%%%%%%%%%%%%%%%%%%%%%%%%%%%%%%%%%%%%%%%%%%%%%%%%%%%

\section{Theorie zur Übung}
	<kurzer Theorieteil - Hinweis auf Besonderheiten>
	

\section{Titel - Übungsaufgabe 1}

	\subsection{Aufgabenstellung}
		<Gegeben ist, Gesucht ist, als Text angeben>
		
	\subsection{Berechnung}
		<Dimensionierung/Berechnung Schaltung, Simulationen, Messbereichsabschätzungen>
		
	\subsection{Messschaltung}
		<Schaltung, eingestellte Messbereiche, BMKZ -> Zuordnung zu Inventarliste>
		
	\subsection{Messvorgang}
		<was ist zu beachten, was/wie wird eingestellt, Einschalt-Reichenfolge, Ausschalt-Reihenfolge>
		
	\subsection{Messtabelle}
		<Messtabelle, Bezeichnung der Mess- Einstellgrößen laut Messschaltung, Einheiten, verwendete Formeln, Beispielrechnung mit Einheiten>
		
	\subsection{Diagramm}
		<Grafische Auswertung, Kommentare zu Besonderheiten/Abweichungen zur Theorie-Erwartung>
		
	\subsection{Ergebnis}
		<verbales Ergebnis / verbale Erkenntnis aus Übung 1>
		
	
\section{Abschließende Bemerkungen}
	<Besondere Vorkommmisse während der Übung (z.B.: Feueralarm, NOT-AUS), besondere Erkenntnisse, Verbesserungsvorschläge>
	

% ACHTUNG: FÜR DEN ANHANG KÖNNEN VON LaTeX LEIDER NICHT DIE RICHTIGEN SEITENZAHLEN ERMITTELT WERDEN!
% DESHALB MÜSSEN DIESE NACH DEM KOMPILIEREN MANUELL IN DER DATEI
% "LA_[Uebungsnummer] - [Uebungstitel]_[jjjj-mm-dd]_[Vorname] [Nachname], [Lfd.Nr.] [Klasse]_Protokoll_[Version].toc"
% ANGEPASST WERDEN! (Dazu muss der Wert in den letzten geschwungenen Klammern der Zeile verändert werden)
% ANSCHLIESSEND MUSS DIE DATEI GESPEICHERT UND DAS PROTOKOLL NEU KOMPILIERT WERDEN

\addcontentsline{toc}{section}{Anhang}                                          % Die section Anhang wird zum Inhaltsverzeichnis hinzugefügt
	\addcontentsline{toc}{subsection}{Datenblatt XY}                            % Wenn keine Datenblätter benötigt werden kann diese Zeile entfernt werden
	\addcontentsline{toc}{subsection}{Messprotokoll (Mitschrift bei der Übung)}
	\addcontentsline{toc}{subsection}{Inventarliste}

%\includepdf[pages={<Seitenzahlen>}]{res/doc/Datenblätter/XY.pdf} % Nur wenn ein Datenblatt eingebunden werden soll
\includepdf[pages=-]{res/doc/\messprotokoll} % Wenn das original Messprotokoll nach dem Ausdrucken dazugeheftet wird MUSS diese Zeile entfernt werden
\includepdf[pages=-]{res/doc/\inventarliste}
