%%%%%%%%%%%%%%%%%%%%%%%%%%%%%%%%%%%%%%%%%%%%%%%%%%%%%%%%%%%%%%%%
% Filename:           macros.tex                               %
% Autor:              Paul Raffer                              %
% Zuletzt bearbeitet: 2020-10-04                               %
% Beschreibung:       In diesem File können die Werte von der  %
%                     Makros festgelegt werden.                %
%%%%%%%%%%%%%%%%%%%%%%%%%%%%%%%%%%%%%%%%%%%%%%%%%%%%%%%%%%%%%%%%

\newcommand{\schulekurzname}             {<Abkürzung der Schule>}
\newcommand{\schulelangname}             {<Name der Schule>}
\newcommand{\schuleabteilung}            {<Abteilung>}
\newcommand{\schulelogo}                 {schule_logo.png}   % Schullogo unter "./res/img/schule_logo.pdf" abspeichern


\newcommand{\uebungstitel}               {<Übungstitel laut Laborplan>}
\newcommand{\uebungsnummer}              {<Übungsnummer>}
\newcommand{\uebungsleiter}              {<Name Übungsleiter>}
\newcommand{\labor}                      {<Labor>}
\newcommand{\nummerderabnahmeeinheit}    {<Nr. Abnahmeeinheit>}
\newcommand{\uebungsdatum}               {<jjjj-mm-dd>}
\newcommand{\abgabedatum}                {<jjjj-mm-dd>}
\newcommand{\klasse}                     {<xABHEL>}
\newcommand{\gruppe}                     {<?a/?b>}
\newcommand{\uebungsteilnehmerIvorname}  {<Vname1>} \newcommand{\uebungsteilnehmerInachname}  {<Nname1>} \newcommand{\uebungsteilnehmerI}  {\uebungsteilnehmerIvorname\ \uebungsteilnehmerInachname}
\newcommand{\uebungsteilnehmerIIvorname} {<Vname2>} \newcommand{\uebungsteilnehmerIInachname} {<Nname2>} \newcommand{\uebungsteilnehmerII} {\uebungsteilnehmerIIvorname\ \uebungsteilnehmerIInachname}
\newcommand{\uebungsteilnehmerIIIvorname}{<Vname3>} \newcommand{\uebungsteilnehmerIIInachname}{<Nname3>} \newcommand{\uebungsteilnehmerIII}{\uebungsteilnehmerIIIvorname\ \uebungsteilnehmerIIInachname}
\newcommand{\uebungsteilnehmerIVvorname} {<Vname4>} \newcommand{\uebungsteilnehmerIVnachname} {<Nname4>} \newcommand{\uebungsteilnehmerIV} {\uebungsteilnehmerIVvorname\ \uebungsteilnehmerIVnachname}
\newcommand{\schriftfuehrervorname}      {<VnameX>} \newcommand{\schriftfuehrernachname}      {<NnameX>} \newcommand{\schriftfuehrer}      {\schriftfuehrervorname\ \schriftfuehrernachname}
\newcommand{\abwesendeuebungsteilnehmer} { & keiner \\ }     % Abwesende Übungsteilnehmer in LaTeX-Tabellenform (& = nächste Spalte; \\ = nächste Zeile) eintragen, wobei die erste Spalte immer leer bleibt und in der zweiten Spalte ein Name eines abwesenden Schülers pro Zeile eingetragen wird
\newcommand{\schueler}                   {\schriftfuehrer}   % Zeile nicht verändern, außer du bist nicht der Schriftführer (z.B. Ersatzprotokoll)


\newcommand{\messprotokoll}              {Messprotokoll.pdf} % Messprotokoll einscannen und unter "./res/doc/Messprotokoll.pdf" abspeichern ODER diese Zeile entfernen und das originale Messprotokoll nach dem Ausdrucken zum Protokoll dazuheften
\newcommand{\inventarliste}              {Inventarliste.pdf} % Inventarliste unter "./res/doc/Inventarliste.pdf" abspeichern


\lhead{\labor}                                             \rhead{\uebungstitel}
\lfoot{\schulekurzname} \cfoot{\schueler\ / \klasse} \rfoot{Seite \thepage\ von \pageref{LastPage}}

\author{\schriftfuehrer}
